Twierdzenie Pitagorasa:
    \[x^2 + y^2 = z^2\] 

Ładny obrazek:
\begin{figure}[h!]
\begin{center}
  \includegraphics[scale=0.25]{pictures/0101-0175.png}
  \caption{Pinkne}
  \label{fig:piękno}
  \end{center}
\end{figure}

\begin{table}
    \caption{Niesamowita tabela}
    \label{fig:tabela mła}

\begin{center}
\begin{tabular}{ |c|c|c| } 
 \hline
 Elemencik1 & Elemencik2 & Elemencik3 \\ 
 coś1 & coś2 & coś3 \\ 
 czegos1 & czegoś2 & czegoś3 \\ 
 \hline
\end{tabular}
\end{center}
\end{table}
\begin{itemize}
    \item Nowy element
    \item Jakiś drugi element
    \item Kolejny z kolei
\end{itemize}

\begin{enumerate}
    \item Pierwsza rzecz
    \item Druga rzecz
    \item ostatnia rzecz
\end{enumerate}

\begin{center}
To jest całkiem nowy akapit w \textbf{Latexie}, (nie Latechu xD). Narzędzia są całkiem \textit{spoko}, nie ma co narzekać. \par
W drugim akapicie opiszę moją historię życia, spotkałem \textsc{kiedyś kolegę} na \underline{przystanku}, było to niesamowite przeżycie, nie wiem czy coś kiedyś je pobije.\par
\end{center}