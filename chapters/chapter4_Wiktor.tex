Odległość punktu P(x, y) od początku układu współrzędnych wynosi:
\[\sqrt{x^2+y^2}\]

\begin{table}[htbp]
    \centering
    \begin{tabular}{|c|c|c|}
        \hline
        p & q & p \textasciicircum q \\ \hline
        0 & 0 & 0                    \\ \hline
        0 & 1 & 0                    \\ \hline
        1 & 0 & 0                    \\ \hline
        1 & 1 & 1                    \\ \hline
    \end{tabular}
    \caption{Tablica prawdy dla koniunkcji}
    \label{tab:koniunkcja}
\end{table}

\begin{figure}[!ht]
    \centering
    \includegraphics[scale=0.2]{pictures/szachy.jpg}
    \caption{Szaszki}
    \label{fig:szaszki}
\end{figure}

\begin{enumerate}
    \item Pierwszy element.
    \item Drugi element.
    \item Trzeci element.
\end{enumerate}

\begin{itemize}
    \item[-] Pierwszy element.
    \item[-] Drugi element.
    \item[-] Trzeci element.
\end{itemize}

\begin{flushleft}
    \textbf{Lorem ipsum} \underline{dolor sit amet}, consectetur adipiscing elit. Vivamus a mauris facilisis, tincidunt sem a, vulputate velit. Rysunek~\ref{fig:szaszki} \par
    \textit{Donec sit amet interdum tortor, vel iaculis libero. Pellentesque accumsan scelerisque libero.} Tabela~\ref{tab:koniunkcja} \par
\end{flushleft}