%\section{Krzysztof Góra}
\label{sec:krzysztof}

Jedynka trygonometryczna
$$\sin^2(x) + \cos^2(x) = 1 $$

\begin{figure}[htbp]
    \centering
    \includegraphics[scale = 0.3]{pictures/dareczek.jpg}
    \caption{Ale to nie do mnie tak, do mnie nie.}
    \label{fig:dareczek}
\end{figure}

% Please add the following required packages to your document preamble:
% \usepackage[table,xcdraw]{xcolor}
% If you use beamer only pass "xcolor=table" option, i.e. \documentclass[xcolor=table]{beamer}
\begin{table}[htbp]
\centering
\caption{Jakieś tam liczby}
\begin{tabular}{|l|l|l|l|}
\hline
Num.1 & Num.2 & Num.3 & Num.4 \\ \hline
21    & 445   & 661   & 5     \\ \hline
997   & 123   & 49    & 12    \\ \hline
16    & 2     & 1287  & 9432  \\ \hline
\end{tabular}
\label{tab:table1_krzysztof}

\end{table}
\centering
Table~\ref{tab:table1_krzysztof} zawiera 4 ciągi liczb

\begin{itemize}
\renewcommand\labelitemi{-}
    \item Ala ma kota
    \item kot ma Ale
    \item a gdzie kucharek sześć
    \item tam nie ma co jeść
\end{itemize}

\begin{enumerate}
    \item Wstęp
    \item Rozwinięcie
    \item Zakończenie
\end{enumerate}

\flushleft
Litwo! \textbf{Ojczyzno moja!} Ty jesteś jak zdrowie,
Ile cię trzeba cenić, ten tylko się dowie,
Kto cię stracił. Dziś piękność twą w całej ozdobie
Widzę i opisuję, bo tęsknię po tobie"
Panno święta, co Jasnej bronisz \textbf{Częstochowy}
I w Ostrej świecisz Bramie! \par Ty, co gród zamkowy
Nowogródzki ochraniasz z jego wiernym ludem!
Jak mnie dziecko do zdrowia powróciłaś cudem,
(Gdy od \emph{płaczącej matki} pod Twoją opiekę
Ofiarowany, martwą podniosłem powiekę
I zaraz mogłem \underline{pieszo} do Twych świątyń progu
Iść za \textit{wrócone życie} podziękować Bogu),
Tak nas powrócisz cudem na Ojczyzny łono.